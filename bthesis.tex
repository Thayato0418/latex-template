%
% bthesis.tex サンプル
%
\documentclass[12pt,a4paper]{jreport}   % 日本語(Japanese thesis)
\usepackage{csis-jbthesis}
\usepackage{url}
\usepackage{makeidx}
\usepackage{cite}
\makeindex
%
% ページスタイルの指定
%
\pagestyle{final}       % 清書
%
% 使用言語の指定
%
\lang{Japanese} % 日本語
%\lang{English} % 英語
%
\doctitle{\bthesis}       % 卒業論文

%
% 日本語題目 (in LaTeX)
%
\title{太陽と月を利用した$\pi$の低速計算アルゴリズムに\\関する理論的研究}
%
% 英語題目 (in LaTeX)
%
\etitle{Theoretical Studies on Low-Speed Calculation Algorithms of $\pi$ \\Utilizing the Sun and the Moon}
%
% 日本語氏名 (in LaTeX)
%   (姓と名の間に空白を入れて下さい)
%
\author{知能 太郎}
%
% 欧文氏名 (in LaTeX)
%   (first name, last name の順に記入し、先頭文字のみを大文字にする。)
%
\eauthor{Taro Chinou}

% 論文提出年月日
%
\syear{2012}
\smonth{2}
\sday{8}

%%講座名
\laboratory{計算機システム第X研究室} %研究室名を記入

%
% 審査委員(日本語)
%   (姓と名、名と称号の間に空白を入れて下さい)
\cmembers{大分 二郎 教授}{指導教員}
%
% 審査委員(英語)
%     (first name, last name の順に記入し、先頭文字のみを大文字にする。
%       first name と last name の間に空白、
%       last name と 称号の間にカンマと空白を入れて下さい。)
%
\ecmembers{Professor Jiro Oita}{Supervisor}
%
% キーワード5~8個 (in LaTeX)
%
\keywords{$\pi$, 天文学, 数学, 計算機, アルゴリズム}
%
% 5 or 8 Keywords (in LaTeX)
%
\ekeywords{$\pi$, astronomy, mathematics, computer, algorithm}

%
% 内容梗概 (in LaTeX)
%
%   注: 行の先頭が\\で始まらないようにすること。
%   800文字程度
\abstract{
人類がこの地上に現われて以来、$\pi$の計算には多くの関心が払われてきた。

本論文では、太陽と月を利用して$\pi$を低速に計算するための
画期的なアルゴリズムを与える。

123456789012345678901234567890
123456789012345678901234567890

ここには内容梗概を書く。ここには内容梗概を書く。ここには内容梗概を書く。
ここには内容梗概を書く。ここには内容梗概を書く。ここには内容梗概を書く。
ここには内容梗概を書く。ここには内容梗概を書く。ここには内容梗概を書く。
ここには内容梗概を書く。ここには内容梗概を書く。ここには内容梗概を書く。
ここには内容梗概を書く。ここには内容梗概を書く。ここには内容梗概を書く。
\index{nicter}
ここには内容梗概を書く。ここには内容梗概を書く。ここには内容梗概を書く。
ここには内容梗概を書く。ここには内容梗概を書く。ここには内容梗概を書く。
ここには内容梗概を書く。ここには内容梗概を書く。ここには内容梗概を書く。
ここには内容梗概を書く。ここには内容梗概を書く。ここには内容梗概を書く。
ここには内容梗概を書く。ここには内容梗概を書く。ここには内容梗概を書く。
}
%
% Abstract (in LaTeX)
%
%   注: 行の先頭が\\で始まらないようにすること。
%   300 word程度
\eabstract{
The calculation of $\pi$ has been paid much attention since human beings
appeared on the earth.

This thesis presents novel low-speed algorithms to calculate
$\pi$ utilizing the sun and the moon.

This is a sample abstract. This is a sample abstract. 
This is a sample abstract. This is a sample abstract. 
This is a sample abstract. This is a sample abstract. 
This is a sample abstract. This is a sample abstract. 
This is a sample abstract. This is a sample abstract. 

This is a sample abstract. This is a sample abstract. 
This is a sample abstract. This is a sample abstract. 
This is a sample abstract. This is a sample abstract. 
This is a sample abstract. This is a sample abstract. 
This is a sample abstract. This is a sample abstract. 
}

%%%%%%%%%%%%%%%%%%%%%%%%% document starts here %%%%%%%%%%%%%%%%%%%%%%%%%%%%
\begin{document}
%
% 表紙 および アブストラクト
%
\titlepage
\firstabstract
\secondabstract
% \thispagestyle{empty}
%
% 目次
%
\pagenumbering{roman}
\toc
\newpage
\listoffigures
% \thispagestyle{empty}
%\newpage
\listoftables
% \thispagestyle{empty}
%
% これ以降本文
%
\setlength{\baselineskip}{23pt}	%行間の設定(消さないこと)
\kanjiskip=.33zw plus 3pt minus 3pt %文字間隔の設定(消さないこと)
\xkanjiskip=.33zw plus 3pt minus 3pt %文字間隔の設定(消さないこと)
\newpage
\pagenumbering{arabic}
\chapter{はじめに}

\section{ほげ}
はじめに はじめに はじめに はじめに はじめに はじめに はじめに はじめに 
はじめに はじめに はじめに はじめに はじめに はじめに はじめに はじめに 
はじめに はじめに はじめに はじめに はじめに はじめに はじめに はじめに 

はじめに はじめに はじめに はじめに はじめに はじめに はじめに はじめに 
はじめに はじめに はじめに はじめに はじめに はじめに はじめに はじめに 
はじめに はじめに はじめに はじめに はじめに はじめに はじめに はじめに 

\ref{kako}節では、過去における研究について述べ、
\ref{kadai}章では、現状と今後の課題について述べる。
また、付録\ref{omake1}におまけその1を添付する。

\section{過去における研究}
\label{kako}

% 過去における研究としては\cite{MWS93}などがある。

123456789012345678901234567890
1234567890123456789012345678901234567890

過去における研究 過去における研究 過去における研究 
過去における研究 過去における研究 過去における研究 過去における研究 
過去における研究 \index{おーぷんふろー@OpenFlow}
過去における研究 過去における研究 過去における研究 
\cite{cryptrec}
過去における研究 過去における研究 過去における研究 過去における研究 
過去における研究 過去における研究 過去における研究 過去における研究 
過去における研究 過去における研究 過去における研究 過去における研究 
過去における研究 過去における研究 過去における研究 過去における研究 
過去における研究 過去における研究 過去における研究 過去における研究 

過去における研究 過去における研究 過去における研究 過去における研究 
過去における研究 過去における研究 過去における研究 過去における研究 
過去における研究 過去における研究 過去における研究 過去における研究 
過去における研究 過去における研究 過去における研究 過去における研究 
過去における研究 過去における研究 過去における研究 過去における研究 

\section{研究の目的と意義}

研究の目的と意義 研究の目的と意義 研究の目的と意義 研究の目的と意義 
研究の目的と意義 研究の目的と意義 研究の目的と意義 研究の目的と意義 
研究の目的と意義 研究の目的と意義 研究の目的と意義 研究の目的と意義 
研究の目的と意義 研究の目的と意義 研究の目的と意義 研究の目的と意義 

研究の目的と意義 研究の目的と意義 研究の目的と意義 研究の目的と意義 
研究の目的と意義 研究の目的と意義 研究の目的と意義 研究の目的と意義 
研究の目的と意義 研究の目的と意義 研究の目的と意義 研究の目的と意義 
研究の目的と意義 研究の目的と意義 研究の目的と意義 研究の目的と意義 

研究の目的と意義 研究の目的と意義 研究の目的と意義 研究の目的と意義 
研究の目的と意義 研究の目的と意義 研究の目的と意義 研究の目的と意義 
研究の目的と意義 研究の目的と意義 研究の目的と意義 研究の目的と意義 
研究の目的と意義 研究の目的と意義 研究の目的と意義 研究の目的と意義 

\begin{figure}
\centerline{ここに図を書く}
\caption{これは図の例}
\end{figure}

\begin{table}
\centerline{ここに表を書く}
\caption{これは表の例}
\end{table}

研究の目的と意義 研究の目的と意義 研究の目的と意義 研究の目的と意義 
研究の目的と意義 研究の目的と意義 研究の目的と意義 研究の目的と意義 
研究の目的と意義 研究の目的と意義 研究の目的と意義 研究の目的と意義 
研究の目的と意義 研究の目的と意義 研究の目的と意義 研究の目的と意義 

研究の目的と意義 研究の目的と意義 研究の目的と意義 研究の目的と意義 
研究の目的と意義 研究の目的と意義 研究の目的と意義 研究の目的と意義 
研究の目的と意義 研究の目的と意義 研究の目的と意義 研究の目的と意義 
研究の目的と意義 研究の目的と意義 研究の目的と意義 研究の目的と意義 

研究の目的と意義 研究の目的と意義 研究の目的と意義 研究の目的と意義 
研究の目的と意義 研究の目的と意義 研究の目的と意義 研究の目的と意義 
研究の目的と意義 研究の目的と意義 研究の目的と意義 研究の目的と意義 
研究の目的と意義 研究の目的と意義 研究の目的と意義 研究の目的と意義 

研究の目的と意義 研究の目的と意義 研究の目的と意義 研究の目的と意義 
研究の目的と意義 研究の目的と意義 研究の目的と意義 研究の目的と意義 
研究の目的と意義 研究の目的と意義 研究の目的と意義 研究の目的と意義 
研究の目的と意義 研究の目的と意義 研究の目的と意義 研究の目的と意義 

研究の目的と意義 研究の目的と意義 研究の目的と意義 研究の目的と意義 
研究の目的と意義 研究の目的と意義 研究の目的と意義 研究の目的と意義 
研究の目的と意義 研究の目的と意義 研究の目的と意義 研究の目的と意義 
研究の目的と意義 研究の目的と意義 研究の目的と意義 研究の目的と意義 

研究の目的と意義 研究の目的と意義 研究の目的と意義 研究の目的と意義 
研究の目的と意義 研究の目的と意義 研究の目的と意義 研究の目的と意義 
研究の目的と意義 研究の目的と意義 研究の目的と意義 研究の目的と意義 
研究の目的と意義 研究の目的と意義 研究の目的と意義 研究の目的と意義 

研究の目的と意義 研究の目的と意義 研究の目的と意義 研究の目的と意義 
研究の目的と意義 研究の目的と意義 研究の目的と意義 研究の目的と意義 
研究の目的と意義 研究の目的と意義 研究の目的と意義 研究の目的と意義 
研究の目的と意義 研究の目的と意義 研究の目的と意義 研究の目的と意義 

研究の目的と意義 研究の目的と意義 研究の目的と意義 研究の目的と意義 
研究の目的と意義 研究の目的と意義 研究の目的と意義 研究の目的と意義 
研究の目的と意義 研究の目的と意義 研究の目的と意義 研究の目的と意義 
研究の目的と意義 研究の目的と意義 研究の目的と意義 研究の目的と意義 

研究の目的と意義 研究の目的と意義 研究の目的と意義 研究の目的と意義 
研究の目的と意義 研究の目的と意義 研究の目的と意義 研究の目的と意義 
研究の目的と意義 研究の目的と意義 研究の目的と意義 研究の目的と意義 
研究の目的と意義 研究の目的と意義 研究の目的と意義 研究の目的と意義 

研究の目的と意義研究の目的と意義研究の目的と意義研究の目的と意義 
研究の目的と意義研究の目的と意義研究の目的と意義研究の目的と意義 
研究の目的と意義研究の目的と意義研究の目的と意義研究の目的と意義 
研究の目的と意義研究の目的と意義研究の目的と意義研究の目的と意義 

研究の目的と意義研究の目的と意義研究の目的と意義研究の目的と意義 
研究の目的と意義研究の目的と意義研究の目的と意義研究の目的と意義 
研究の目的と意義研究の目的と意義研究の目的と意義研究の目的と意義 
研究の目的と意義研究の目的と意義研究の目的と意義研究の目的と意義 

研究の目的と意義研究の目的と意義研究の目的と意義研究の目的と意義 
研究の目的と意義研究の目的と意義研究の目的と意義研究の目的と意義 
研究の目的と意義研究の目的と意義研究の目的と意義研究の目的と意義 
研究の目的と意義研究の目的と意義研究の目的と意義研究の目的と意義 


\newpage

\chapter{提案手法}\label{proposal}

\section{$\pi$の高速計算手法}
スパコンで$\pi$計算をぶんまわします。

\subsection{アルゴリズム}
\subsubsection{$\pi$高速化アルゴリズム}

\newpage
\chapter{評価}\label{eval}
\section{実験方法}
既存手法と計算時間を比較します。
\subsection{実験(1)}
\subsection{実験(2)}

\newpage
\chapter{現状と今後の課題}\label{kadai}

現状と今後の課題 現状と今後の課題 現状と今後の課題 現状と今後の課題 
現状と今後の課題 現状と今後の課題 現状と今後の課題 現状と今後の課題 
現状と今後の課題 現状と今後の課題 現状と今後の課題 現状と今後の課題 
現状と今後の課題 現状と今後の課題 現状と今後の課題 現状と今後の課題 

現状と今後の課題 現状と今後の課題 現状と今後の課題 現状と今後の課題 
現状と今後の課題 現状と今後の課題 現状と今後の課題 現状と今後の課題 
現状と今後の課題 現状と今後の課題 現状と今後の課題 現状と今後の課題 
現状と今後の課題 現状と今後の課題 現状と今後の課題 現状と今後の課題 

現状と今後の課題 現状と今後の課題 現状と今後の課題 現状と今後の課題 
現状と今後の課題 現状と今後の課題 現状と今後の課題 現状と今後の課題 
現状と今後の課題 現状と今後の課題 現状と今後の課題 現状と今後の課題 
現状と今後の課題 現状と今後の課題 現状と今後の課題 現状と今後の課題 
\index{OpenFlowあーきてくちゃ@OpenFlowアーキテクチャ}
%
% 謝辞
%
\acknowledgements

Thank you. Thank you.
%
% 参考文献
% ここでは \reference を使って、自分でリストを作るか、BibTeX を使って
% リストをつくって下さい。この例では BibTeX を作るような形式になってい
% ます。
%
\newpage
% \reference
\bibliography{bunken}
\bibliographystyle{cs2srt}
%
% 付録
%
\appendix

\section{おまけその1}
\label{omake1}

これはおまけです。これはおまけです。これはおまけです。これはおまけです。
これはおまけです。これはおまけです。これはおまけです。これはおまけです。
これはおまけです。これはおまけです。これはおまけです。これはおまけです。
これはおまけです。これはおまけです。これはおまけです。これはおまけです。

\begin{figure}
\centerline{これはおまけの図です。}
\caption{おまけの図}
\end{figure}


\section{おまけその2}

これもおまけです。これもおまけです。これもおまけです。これもおまけです。
これもおまけです。これもおまけです。これもおまけです。これもおまけです。
これもおまけです。これもおまけです。これもおまけです。これもおまけです。
これもおまけです。これもおまけです。これもおまけです。これもおまけです。

\end{document}

